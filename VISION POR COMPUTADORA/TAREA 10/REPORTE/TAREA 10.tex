% !TeX TXS-program:compile = txs:///pdflatex/[--shell-escape]

\documentclass[11pt, letterpaper]{article}

\usepackage{minted}
\usepackage[utf8]{inputenc}
\usepackage[T1]{fontenc}
\usepackage{lmodern}
\usepackage{graphicx}
\usepackage{longtable}
\usepackage{wrapfig}
\usepackage{rotating}
\usepackage{amsmath}
\usepackage{textcomp}
\usepackage{amssymb}
\usepackage{hyperref}
\usepackage[round]{natbib}
\usepackage{subcaption}


\title{\bfseries Tarea}
\author{Ángel García Báez}
\date{\today}
\setcounter{tocdepth}{3} 

\begin{document}
	
	% Página de presentación
	\begin{titlepage}
		\centering
		\includegraphics[width=0.2\textwidth]{logo.png}\par
		\vspace{1cm}
		{\LARGE \bfseries Universidad Veracruzana \par}
		\vspace{1cm}
		{\Large Maestría en Inteligencia Artificial\par}
		\vspace{3cm}
		{\LARGE \bfseries Visión por Computadora \par}
		\vspace{1cm}
		{\Large \bfseries Tarea 10. Propuesta de proyecto para visión por computadora y lógica difusa. \par}
		\vfill
		{\Large \textit{Ángel García Báez}\par}
		\vspace{1cm}
		{\Large Profesor: Dr. Héctor Acosta Mesa \par}
		\vfill
		{\Large \today \par}
	\end{titlepage}
		
% Sección para el problema 1
\section{Propuesta de proyecto.}
\subsection{Comparativa entre K-means y Fuzzy C-means para la agrupación de caracteristicas en imagenes de rostros.}


Se propone realizar una comparativa entre el desempeño de los algoritmos K-means y Fuzzy C-means para la agrupación de imágenes en un contexto no supervisado. Se plantea el uso de una red neuronal convolucional (CNN) pre-entrenada como extractor de características, generando un vector representativo por imagen. Estos vectores conforman una matriz homogénea de tamaño  N×P, donde N es el número de imágenes y P la cantidad de características extraídas. Sobre esta matriz se aplican ambos métodos de agrupamiento, su rendimiento se evalúa mediante la intra-varianza, como medida de compactación de los grupos. El objetivo es determinar cuál de los enfoques logra una mejor segmentación del espacio de características generado por la CNN.


	
	
	
\end{document}

