% !TeX document-id = {1700c40c-d810-430a-9d0b-2789b57637de}
% !TeX TXS-program:compile = txs:///pdflatex/[--shell-escape]

\documentclass[11pt, letterpaper]{article}

\usepackage[utf8]{inputenc}
\usepackage[T1]{fontenc}
\usepackage{lmodern}
\usepackage{graphicx}
\usepackage{longtable}
\usepackage{wrapfig}
\usepackage{rotating}
\usepackage{amsmath}
\usepackage{textcomp}
\usepackage{amssymb}
\usepackage{hyperref}
\usepackage{minted}
\usepackage[spanish]{babel}
\usepackage[round]{natbib}
\usepackage{logicproof}
\usepackage{enumerate}%

\title{\textsc{Representación del conocimiento} \\
	Tarea 2
}  

\author{Angel García Báez\\
	Alumno de la Maestría en Inteligencia Artificial \\ \\ \textbf{IIIA}
	Instituto de Investigaciones en Inteligencia Artificial \\
	Universidad Veracruzana \\ \emph{Campus Sur, Calle Paseo Lote II,
		Sección 2a, No 112} \\ \emph{Nuevo Xalapa, Xalapa, Ver., México 91097}
	\\ \\ ZS24019400@estudiantes.uv.mx}

\date{\today}


\begin{document}
	
	\maketitle

	\newpage
	
	\section{Logica proposicional en base al reglamento de posgrados}
	
	Con base en el Reglamento General de Estudios de Posgrado de nuestra
	universidad, defina en lógica proposicional los requisitos que debe cumplir la
	persona para presentar el examen para obtener su grado académico (Artículo
	65)[30/100] \\
	
	Se hizo la consulta del Articulo 65 del Reglamento General de Estudios de Posgrado de la Universidad Veracruzana. Dicho articulo abarca los siguientes requisitos para poder presentar el examen de la obtención del diploma o grado académico que deben cumplir los estudiantes:
	
	\begin{enumerate}[I]
		\item Haber acreditado todas las experiencias educativas y actividades academicas que establezca el plan de estudios del programa educativo correspondiente;		
		\item Presentar el certificado de estudio del posgrado cursado;
		\item Presentar los votos aprobatorios de los sinodales;
		\item Presentar la carta responsiva anexa a este reglamento, firmada por el director del trabajo recepcional;
		\item Aval del Consejo Tecnico u organo equivalente, en caso de haber excedido el tiempo reglamentario para la presentación del examen;
		\item No tener adeudos con la Universidad Veracruza;
		\item Pagar el arancel correspondiente; y
		\item Las demas que señale la normatividad universitaria.		
	\end{enumerate}
	
	\newpage 
	
	En un primer intento por descomponer la normativa para pasarla a lenguaje de logica proposicional, se hizo una lectura detenida de las condiciones para presentar el examen, dicho primer intento busca convertir los enunciados en variables logicas como se muestra:
	
	\begin{enumerate}[I]
	\item Haber acreditado todas las experiencias educativas ($a$) y ($\wedge$) actividades academicas que establezca el plan de estudios del programa educativo correspondiente ($b$);
	
	\item Presentar el certificado de estudio del posgrado cursado ($c$);
	\item Presentar los votos aprobatorios de los sinodales ($d$);
	\item Presentar la carta responsiva anexa a este reglamento, firmada por el director del trabajo recepcional; ($e$)
	\item Aval del Consejo Tecnico ($f$) u ($\vee$) organo equivalente ($g$), en caso de haber excedido el tiempo reglamentario para la presentación del examen ($h$);
	\item No tener adeudos con la Universidad Veracruza; ($i$)
	\item Pagar el arancel correspondiente; y ($j$)
	\item Las demas que señale la normatividad universitaria. ($k$).
	\item Si se cumplen con las condiciones requeridas, implica que se lleva a cabo la presentación del examen. ($l$).
	\end{enumerate}

	Identificadas las posibles variables, se re escribe una versión simplificada antes de pasarla totalmente a logica proposicional: \\
	
	Si $a$ se cumple y $b$ se cumple junto con $c$, $d$ y $e$. Ademas, si $h$ se presenta, entonces se debe cumplir $f$ o $g$. Junto con todo lo demas, es indispensable que se cumpla $i$, $j$, y $k$. Si todo se cumple adecuadamente, entonces se lleva a cabo $l$ \\
	
	Pasandolas en forma de logica proposicional:
	
	$$((a \wedge b) \wedge c \wedge d \wedge e \wedge (h \implies (f \vee g)) \wedge i \wedge j \wedge k) \implies l$$
	
	
	
	
	
	
	
	
	
	
	
	
	
	
	\newpage
	
	\section{Implementación del algoritmo CNF en prolog}
	
	Implemente en Prolog el algoritmo CNF visto en clase, para convertir una
	fbf proposicional en su equivalente en forma normal conjuntiva. Pruebe su
	implementación con el ejemplo visto en clase.[30/100]
		
	\newpage
	
	\section{Conversión de logica proposicional a CNF}
	
	Convierta los requisitos del ejercicio uno a forma normal conjuntiva, usando
	su programa CNF. [20/100] \\
	
	\newpage

	\section{Conversión de logica proposicional a CNF}

	Utilice los algoritmos CNF y SAT para verifivar que $p \implies q$ es equivalente a $\neg p \vee q$ (Ejercicio 3 de la tarea anterior). [20/100] \\


		
\newpage


\section{Referencias}  % Sección numerada de referencias
\bibliographystyle{apalike}  % Estilo de citas (puedes cambiarlo)
\bibliography{Biblio}        % Nombre del archivo BibTeX (sin extensión)


	
	
\end{document}


\end{document}
