% !TeX document-id = {1700c40c-d810-430a-9d0b-2789b57637de}
% !TeX TXS-program:compile = txs:///pdflatex/[--shell-escape]

\documentclass[11pt, letterpaper]{article}

\usepackage[utf8]{inputenc}
\usepackage[T1]{fontenc}
\usepackage{lmodern}
\usepackage{graphicx}
\usepackage{longtable}
\usepackage{wrapfig}
\usepackage{rotating}
\usepackage{amsmath}
\usepackage{textcomp}
\usepackage{amssymb}
\usepackage{hyperref}
\usepackage{minted}
\usepackage[spanish]{babel}
\usepackage[round]{natbib}
\usepackage{logicproof}


\title{\textsc{Representación del conocimiento} \\
	Tarea 2
}  

\author{Angel García Báez\\
	Alumno de la Maestría en Inteligencia Artificial \\ \\ \textbf{IIIA}
	Instituto de Investigaciones en Inteligencia Artificial \\
	Universidad Veracruzana \\ \emph{Campus Sur, Calle Paseo Lote II,
		Sección 2a, No 112} \\ \emph{Nuevo Xalapa, Xalapa, Ver., México 91097}
	\\ \\ ZS24019400@estudiantes.uv.mx}

\date{\today}


\begin{document}
	
	\maketitle

	\newpage
	
	\section{Logica proposicional en base al reglamento de posgrados}
	
	Con base en el Reglamento General de Estudios de Posgrado de nuestra
	universidad, defina en lógica proposicional los requisitos que debe cumplir la
	persona para presentar el examen para obtener su grado académico (Artículo
	65)[30/100] \\
	
	
	\newpage
	
	\section{Implementación del algoritmo CNF en prolog}
	
	Implemente en Prolog el algoritmo CNF visto en clase, para convertir una
	fbf proposicional en su equivalente en forma normal conjuntiva. Pruebe su
	implementación con el ejemplo visto en clase.[30/100]
		
	\newpage
	
	\section{Conversión de logica proposicional a CNF}
	
	Convierta los requisitos del ejercicio uno a forma normal conjuntiva, usando
	su programa CNF. [20/100] \\
	
	\newpage

	\section{Conversión de logica proposicional a CNF}

	Utilice los algoritmos CNF y SAT para verifivar que $p \implies q$ es equivalente a $\neg p \vee q$ (Ejercicio 3 de la tarea anterior). [20/100] \\


		
\newpage


\section{Referencias}  % Sección numerada de referencias
\bibliographystyle{apalike}  % Estilo de citas (puedes cambiarlo)
\bibliography{Biblio}        % Nombre del archivo BibTeX (sin extensión)


	
	
\end{document}


\end{document}
