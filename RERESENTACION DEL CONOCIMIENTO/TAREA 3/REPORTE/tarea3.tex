% !TeX document-id = {1700c40c-d810-430a-9d0b-2789b57637de}
% !TeX TXS-program:compile = txs:///pdflatex/[--shell-escape]

\documentclass[11pt, letterpaper]{article}

\usepackage[utf8]{inputenc}
\usepackage[T1]{fontenc}
\usepackage{lmodern}
\usepackage{graphicx}
\usepackage{longtable}
\usepackage{wrapfig}
\usepackage{rotating}
\usepackage{amsmath}
\usepackage{textcomp}
\usepackage{amssymb}
\usepackage{hyperref}
\usepackage{minted}
\usepackage[spanish]{babel}
\usepackage[round]{natbib}
\usepackage{logicproof}
\usepackage{enumerate}%

\title{\textsc{Representación del conocimiento} \\
	Tarea 3
}  

\author{Angel García Báez\\
	Alumno de la Maestría en Inteligencia Artificial \\ \\ \textbf{IIIA}
	Instituto de Investigaciones en Inteligencia Artificial \\
	Universidad Veracruzana \\ \emph{Campus Sur, Calle Paseo Lote II,
		Sección 2a, No 112} \\ \emph{Nuevo Xalapa, Xalapa, Ver., México 91097}
	\\ \\ ZS24019400@estudiantes.uv.mx}

\date{\today}


\begin{document}
	
\maketitle

\newpage
	
\section{Representación en Prolog del universo del discurso y las relaciones entre objetos}
	
Con base en el lineamiento estandarizado para la vigilancia epidemiológica y
por laboratorio de COVID-19:
	
Represente en Prolog el universo de discurso y la relaciones entre objetos
pertinentes para las páginas 11-17. (50/100).	
	
	
	
\newpage
	
\section{Ejemplos de cada caso de las reglas terminológicas vistas en clases.}
	
	Ejemplifique cada caso de regla terminológica vista en clase (rc-slides-05,
	77) con las definiciones de su representación. (20/100).
	

\newpage
	
\section{Ejemplos de las preguntas que puede resolver la representación}
	
	Ejemplifique y justifique el tipo de preguntas que su representación y Prolog
	pueden resolver. (30/100) \\
	

		
\newpage

\section{Referencias}  % Sección numerada de referencias
\bibliographystyle{apalike}  % Estilo de citas (puedes cambiarlo)
\bibliography{Biblio}        % Nombre del archivo BibTeX (sin extensión)

	
\end{document}
 

