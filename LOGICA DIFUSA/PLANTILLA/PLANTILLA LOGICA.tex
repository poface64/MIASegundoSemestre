\documentclass[11pt, letterpaper]{article}

\usepackage[utf8]{inputenc}
\usepackage[T1]{fontenc}
\usepackage{lmodern}
\usepackage{graphicx}
\usepackage{longtable}
\usepackage{wrapfig}
\usepackage{rotating}
\usepackage{amsmath}
\usepackage{textcomp}
\usepackage{amssymb}
\usepackage{hyperref}
\usepackage[spanish]{babel}
\usepackage[round]{natbib}

\title{\bfseries Tarea}
\author{Ángel García Báez}
\date{\today}

\begin{document}
	
	% Página de presentación
	\begin{titlepage}
		\centering
		\includegraphics[width=0.2\textwidth]{logo.png}\par
		\vspace{1cm}
		{\LARGE \bfseries Universidad Veracruzana \par}
		\vspace{1cm}
		{\Large Maestría en Inteligencia Artificial\par}
		\vspace{3cm}
		{\Large \bfseries Tarea 1. Problema del mesero y problema del confort con logica difusa en matlab. \par}
		\vfill
		{\Large \textit{Ángel García Báez}\par}
		\vfill
		{\Large \today \par}
	\end{titlepage}
	
	% Página exclusiva para la tabla de contenidos
	\newpage
	\tableofcontents
	\newpage
	
	% Secciones del documento
	\section{Introducción}
	Aquí puedes comenzar a escribir la introducción del trabajo.
	
	\section{Problema}
	Descripción del problema abordado en la tarea.
	
	\section{Experimentos}
	Detalles de los experimentos realizados.
	
	\section{Conclusiones}
	Conclusiones derivadas del trabajo.
	
\end{document}
